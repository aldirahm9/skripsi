%!TEX root = ./template-skripsi.tex
%-------------------------------------------------------------------------------
% 								BAB I
% 							LATAR BELAKANG
%-------------------------------------------------------------------------------

\chapter{PENDAHULUAN}

\section{Latar Belakang Masalah}
	Tri Dharma Perguruan Tinggi adalah tiga pilar dasar pemikiran yang harus ada pada semua aspek di dalam sebuah perguruan tinggi mulai dari mahasiswa, dosen dan berbagai civitas akademika yang terlibat. Tri Dharma Perguruan Tinggi terdiri dari 3 poin, yaitu Pendidikan dan Pengajaran, Penelitian dan Pengembangan, dan Pengabdian Kepada Masyarakat.

	Kegiatan belajar mengajar adalah suatu kegiatan yang merupakan bagian dari salah satu isi Tri Dharma Perguruan Tinggi yang merupakan pendidikan dan pengajaran. Kegiatan belajar mengajar dijalankan oleh dua pihak, yaitu dosen dan mahasiswa. Keduanya merupakan faktor yang sangat berpengaruh dalam berhasilnya suatu kegiatan belajar mengajar. Dosen dan mahasiswa merupakan komponen penting dalam suatu sistem pembelajaran di perguruan tinggi dimana peran, tugas, dan tanggung jawab seorang dosen terutama dalam sebuah proses pembelajaran merupakan hal yang sangat penting dalam mewujudkan tujuan dari satu pilar Tri Dharma Perguruan Tinggi. Dalam melaksanakan kegiatan belajar mengajar perlu dilakukan pemantauan dan evaluasi untuk memastikan pelaksanaan kegiatan belajar mengajar mencapai tujuan yang diinginkan dan sesuai dengan standar yang ditetapkan oleh perguruan tinggi.

	Monitoring merupakan kegiatan pemantauan yang dilakukan untuk mengetahui proses kegiatan Belajar Mengajar yang dilakukan oleh dosen. Sedangkan evaluasi merupakan hasil akhir dari monitoring yang dilakukan selama proses Kegiatan Belajar Mengajar yang telah dilaksanakan selama satu semester. Monitoring dan evaluasi kegiatan belajar mengajar adalah tugas dari Tim Penjamin Mutu. Kegiatan ini dilakukan dengan tujuan untuk memastikan Kegiatan Belajar Mengajar yang disampaikan oleh dosen menerapkan aturan dan standar yang telah diterapkan oleh perguruan tinggi serta sesuai dengan Rencana Pembelajaran Semester (RPS).

	Seiring berkembangnya teknologi, sistem informasi dapat mempermudah berbagai kegiatan untuk menghasilkan informasi sebagai penunjang pengambilan keputusan dan mempermudah penyelesaian suatu masalah dan meningkatkan kinerja berbagai aktivitas. Penggunaan sistem informasi juga dapat dimanfaatkan pada berbagai macam aktivitas akademik pada sebuah perguruan tinggi seperti aktivitas kegiatan belajar mengajar beserta monitoring dan evaluasinya.

	Pandemi COVID-19 yang terjadi pada tahun 2020 dan masih berjalan hingga 2021 ini juga menyebabkan berubahnya kegiatan belajar mengajar yang sebelumnya dilakukan tatap muka di dalam ruang kelas menjadi daring atau \textit{online}. Dilakukannya kegiatan belajar mengajar menjadi daring menyebabkan sistem presensi dan pemantauan kelas manual yang menggunakan kertas tidak dapat lagi digunakan sehingga universitas yang masih menggunakan sistem manual terpaksa menggunakan solusi daring sementara seperti Google Form, Microsoft Form, atau tetap melakukan presensi dengan manual menggunakan file excel. Penggunaan form daring tersebut tentunya tidak efisien dikarenakan form-form tersebut tidak terintegrasi dalam satu sistem sehingga dibutuhkan suatu sistem informasi yang dapat memfasilitasi jalannya kegiatan belajar mengajar daring dan sekaligus mempermudah pemantauan kegiatan belajar mengajar tersebut.

	Pada Fakultas Matematika dan Ilmu Pengetahuan Alam (FMIPA) UNJ, kegiatan belajar mengajar dipantau melalui formulir 05 dan 06 oleh Tim Penjamin Mutu tingkat Program Studi yang selanjutkan diserahkan kepada Gugus Penjamin Mutu FMIPA UNJ. Dalam menjalankan kegiatan belajar mengajar, mahasiswa akan mengambil formulir 05 dan 06 dari ruang program studi setiap perkuliahan akan dimulai dan akan dikembalikan ke ruang program studi ketika perkuliahan selesai. Formulir 05 dan 06 akan dievaluasi pada awal, pertengahan dan akhir semester oleh Tim Penjamin Mutu program studi. Evaluasi pada awal semester dilakukan untuk memantau jalannya perkuliahan pertama dan melihat apakah perkuliahan berjalan sesuai dengan jadwal akademik. Evaluasi kedua dilakukan setelah pertemuan ke-8 yang merupakan pertemuan Ujian Tengah Semester (UTS). Evaluasi tersebut dilakukan untuk memastikan semua pertemuan sebelumnya sudah lengkap sebelum melakukan UTS. Evaluasi terakhir dilakukan setelah pertemuan ke-16 dan dilakukan untuk melihat jumlah seluruh pertemuan mencapai target minimal, yaitu 80\% pertemuan. Namun karena sistem yang masih manual dan formulir yang menggunakan kertas, proses evaluasi memakan waktu yang cukup lama karena pihak Tim Penjamin Mutu harus mengecek setiap formulir 05 dan 06.

	Formulir 05 dan 06 merupakan formulir yang berisi tentang bagaimana kegiatan belajar mengajar dilaksanakan oleh seorang dosen dan mahasiswa. Dimana formulir 05 berisi tentang materi yang disampaikan oleh dosen dan formulir 06 berisi presensi mahasiswa dan nilai-nilai yang diberikan oleh dosen kepada mahasiswa. Nilai-nilai tersebut terdiri dari nilai-nilai tugas, Ujian Tengah Semester (UTS), dan Ujian Akhir Semester (UAS). Setelah adanya SIAKAD (Sistem Informasi Akademik) UNJ, dosen-dosen tidak lagi mengisi kolom nilai pada formulir 06 melainkan langsung memasukkan nilai akhir di SIAKAD pada saat akhir semester. Tidak diisinya kolom nilai pada formulir 06 menyebabkan transparansi nilai yang berkurang untuk pihak mahasiswa.

	Hasil penelitian yang telah dilakukan oleh \cite{FitriAndiniMedIrzal2017} dalam jurnal yang berjudul "Perancangan dan Implementasi Sistem Absensi \textit{Online} Berbasis Android di Lingkungan Universitas Negeri Jakarta" menunjukkan bahwa sistem presensi online dapat diterapkan menjadi salah satu cara agar proses presensi mahasiswa dapat berlangsung secara cepat dan membuat data presensi menjadi semakin terstruktur. Pada penelitian lain oleh \cite{Kultsum2021} yang melakukan penelitian berjudul "Rancang Bangun Sistem Presensi Akademik Berbasis Web Dengan \textit{Framework} Laravel di Lingkungan Program Studi Ilmu Komputer Universitas Negeri Jakarta" menunjukkan bahwa penggunaan aplikasi web memiliki beberapa kelebihan seperti kemudahan akses, kemudahan perawatan, dan kebutuhan perangkat keras yang lebih rendah dan tetap dapat diterapkan sebagai opsi untuk mengurangi masalah yang terjadi pada sistem manual.

	Pada penelitian ini, penulis mengembangkan dari penelitian-penelitian sebelumnya yang hanya berupa sistem presensi dengan tambahan fitur monitoring untuk mempermudah tugas Tim Penjamin Mutu, kelengkapan form 06 berupa pengisian nilai untuk memberikan transparansi nilai kepada mahasiswa dan membuat data nilai menjadi lebih terstruktur. Penulis memilih membuat sistem berbasis \textit{website} agar dapat lebih mudah diakses tanpa membedakan \textit{device} yang digunakan. Dalam pembuatan sistem informasi ini, penulis menggunakan metode Spiral sebagai metode pengembangannya karena penerapannya yang cukup mudah dan merupakan metode yang fleksibel jika terjadi perubahan pada sistem.

	Dengan adanya sebuah sistem informasi monitoring yang dapat membantu proses monitoring kegiatan belajar mengajar akan sangat memudahkan Tim Penjamin Mutu untuk melakukan monitor dan evaluasi kegiatan belajar mengajar tanpa memakan waktu yang cukup lama, mempermudah pengisian form 05 dan 06 pada perkuliahan dengan tidak menggunakan form manual berbentuk kertas dan memberikan mahasiswa transparansi nilai yang diberikan oleh dosen.

	
\section{Identifikasi Masalah}
Berdasarkan latar belakang di atas, masalah yang akan diidentifikasi adalah sebagai berikut: 
\begin{enumerate}
	\item Pengisian form 05 dan 06 dalam proses pembelajaran masih manual, sehingga monitoring yang dilakukan kurang efisien karena masih melihat kertas form satu persatu.
	\item Kurangnya transparansi nilai yang diberikan dosen kepada mahasiswa.
\end{enumerate}

\section{Pembatasan Masalah}
Pada perancangan sistem ini, penulis membatasi masalah sebagai berikut:
\begin{enumerate}
	\item Sistem ini hanya dibuat untuk digunakan pada Rumpun Matematika FMIPA UNJ.
	\item Implementasi sistem di jaringan lokal.	`
\end{enumerate}

\section{Rumusan Masalah}
Berdasarkan uraian pada latar belakang yang diutarakan di atas, maka perumusan masalah pada penelitian ini adalah “Bagaimana merancang suatu sistem informasi monitoring kegiatan belajar mengajar menggunakan \textit{web service} SIAKAD di Rumpun Matematika FMIPA UNJ?".


\section{Tujuan Penelitian}
Tujuan pada penelitian ini yaitu merancang dan membangun Sistem Informasi Monitoring Kegiatan Belajar Mengajar Menggunakan Web Service SIAKAD di Rumpun Matematika FMIPA UNJ.

\section{Manfaat Penelitian}
Hasil penelitian ini diharapkan dapat memberikan manfaat bagi berbagai pihak, di antaranya:
\begin{enumerate}
	\item Bagi Mahasiswa 
		
	Sebagai suatu media untuk memudahkan memantau nilai yang didapat pada setiap mata kuliah secara transparan.
	
	\item Bagi Dosen 
	 	
	Mampu mempermudah dosen untuk melakukan pengisian form 05 dan 06 dan memantau kehadiran mahasiswanya.

	\item Bagi Tim Penjamin Mutu

	Mempermudah monitoring kegiatan belajar mengajar untuk melakukan evaluasi kinerja dosen.

	\item Bagi Program Studi

	Mempermudah evaluasi kehadiran dosen dan mahasiswa dalam rangka menyusun borang akreditasi program studi.
\end{enumerate}


% Baris ini digunakan untuk membantu dalam melakukan sitasi
% Karena diapit dengan comment, maka baris ini akan diabaikan
% oleh compiler LaTeX.
\begin{comment}
\bibliography{daftar-pustaka}
\end{comment}
