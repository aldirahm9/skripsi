%!TEX root = ./template-skripsi.tex
%-------------------------------------------------------------------------------
%                            	BAB IV
%               		KESIMPULAN DAN SARAN
%-------------------------------------------------------------------------------

\chapter{KESIMPULAN DAN SARAN}

\section{Kesimpulan}
Berdasarkan hasil implementasi dan pengujian fitur pada sistem informasi yang telah dibangun, maka dapat diambil kesimpulan sebagai berikut:

\begin{enumerate}
	\item Sistem Informasi Monitoring Kegiatan Belajar Mengajar menggunakan \textit{web service} SIAKAD di rumpun matematika FMIPA Universitas Negeri Jakarta merupakan sebuah pengembangan dari penelitian \cite{Kultsum2021} dengan penambahan fituryang belum ada sebelumnya yaitu monitoring Tim Penjamin Mutu dan penilaian tugas.

	\item  Pengembangan sistem dilakukan dengan menggunakan metode pengembangan perangkat lunak SDLC (\textit{System Development Life Cycle}) dengan model \textit{spiral}. Pada rancangan awal jumlah iterasi spiral yang akan dilakukan sebanyak tiga iterasi, namun karena adanya tambahan kebutuhan sistem, maka dilakukan satu iterasi tambahan sehingga total iterasi menjadi empat iterasi.
	
	%\item Sistem Informasi Monitoring Kegiatan Belajar Mengajar dibangun dengan menggunakan \textit{framework} Laravel pada \textit{backend} dan \textit{Vue} pada \textit{frontend}.
	
	\item  Berdasarkan hasil pengujian UAT (\textit{User Acceptance Test}) pada sisi fungsional, didapatkan nilai 100\% sehingga dapat ditarik kesimpulan bahwa fitur-fitur yang ada pada Sistem Informasi Monitoring Kegiatan Belajar Mengajar dapat berjalan dengan baik dan sesuai kebutuhan.
	
	\item Berdasarkan hasil pengujian \textit{User Acceptance Test} pada sisi kebergunaan (\textit{usabilty}), didapatkan penilaian 91.8\% untuk total persentase kelayakan dari keseluruhan sistem. Penilaian tersebut pada skala likert masuk pada skor skala 81\%-100\%, maka dapat dikatakan nilai kebergunaan dari Sistem Informasi Monitoring Kegiatan Belajar Mengajar mendapatkan predikat sangat sesuai.
\end{enumerate}

\section{Saran}
Adapun saran untuk penelitian selanjutnya adalah:
\begin{enumerate} 
	\item Meningkatkan ruang lingkup penelitian dari rumpun menjadi fakultas.
	\item Mengintegrasikan Sistem Informasi Monitoring Kegiatan Belajar Mengajar dengan sistem \textit{e-learning} untuk langsung mendapatkan nilai tugas dari \textit{e-learning}.

\end{enumerate}


% Baris ini digunakan untuk membantu dalam melakukan sitasi
% Karena diapit dengan comment, maka baris ini akan diabaikan
% oleh compiler LaTeX.
\begin{comment}
\bibliography{daftar-pustaka}
\end{comment}